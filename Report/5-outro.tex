\chapter{Заключение}
В ходе данной работы была разработана автоматической тестирующей системы. Пользователь имеет возможность создавать и проходить тесты.

Определены роли пользователей: администратор, создает тесты, и испытуемые, выполняет задания. 

Детально были описаны процесса авторизации, регистрации, добавления и прохождения тестов с использование IDEF0 диаграмм. 

Выявлены следующие сущности: \texttt{Пользователь}, \texttt{Тест}, \texttt{Результат теста}, \texttt{Вопрос}. Пользователь создает и проходит тесты. Тесты содержат вопросы. В результате выполнения теста, пользователь получает результат теста. Для данных сущностей построена ER-диаграмма. Описаны и документированы атрибуты каждой сущности.

С помощью диаграмм последовательностей описаны взаимодействия системы, в следующих действиях: авторизация, регистрация, запрос данных пользователя, изменение имени пользователя, изменение пароля пользователя, изменение электронной почты пользователя, добавление нового теста, прохождение теста, удаление теста, получение списка созданных тестов, поиск тестов.

В качестве базы данных выбрана нерялиционная документооринетированная база данных MongoDB. Были описаны преимуществе данной БД.

Для разработки серверной части использовался язык Haskell. В качестве веб-фреймворка использовалась библиотека Servant. Пакет Persistent использовался для взаимодействия с базой данных. Клиентская часть разработана на языке Elm.
